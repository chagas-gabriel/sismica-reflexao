\documentclass[a4paper,12pt]{article}
\usepackage[utf8]{inputenc}
\usepackage[T1]{fontenc}
\usepackage[brazil]{babel}
\usepackage{geometry}
\usepackage{amsmath}
\usepackage{graphicx}
\usepackage{enumitem}
\usepackage{xcolor}
\usepackage{listings}
\usepackage{float}
\usepackage{indentfirst}


% Configuração das margens
\geometry{top=2.5cm, bottom=2.5cm, left=2.5cm, right=2.5cm}

% Configuração para exibição de comandos de terminal
\lstset{
    basicstyle=\ttfamily\small,
    breaklines=true,
    frame=single,
    backgroundcolor=\color{gray!10},
    columns=fullflexible
}

\begin{document}

% --- Cabeçalho ---
\noindent
\textbf{AGG0327 Sísmica de Reflexão - Atividade 4} \\
\textbf{Nome(s):} Gabriel Aparecido das Chagas Silva \\
\textbf{Data:} 25/11/2025

\vspace{0.5cm}

% --- PARTE I ---
\section*{PARTE I: Migração no domínio da frequência: exemplo didático com dado sintético}

\textbf{Objetivos:} efetuar a migração sobre uma seção empilhada e analisar a influência de alguns fatores na qualidade da migração realizada no domínio da frequência.

\subsection*{Arquivos de dados sintéticos:}

\begin{itemize}
    \item \texttt{data10.5.dx40}: frequência dominante da wavelet = 10 Hz; comprimento do registro ($T_{max}$) = 5 s; e espaçamento entre cdps = 40 m.
    \item \texttt{data10.8.dx40}: frequência dominante da wavelet = 10 Hz; comprimento do registro ($T_{max}$) = 8 s; e espaçamento entre cdps = 40 m.
    \item \texttt{data10.8.dx20}: frequência dominante da wavelet = 10 Hz; comprimento do registro ($T_{max}$) = 8 s; e espaçamento entre cdps = 20 m.
    \item \texttt{data20.8.dx40}: frequência dominante da wavelet = 20 Hz; comprimento do registro ($T_{max}$) = 8 s; e espaçamento entre cdps = 40 m.
\end{itemize}

\hrule
\vspace{0.5cm}

\begin{enumerate}[label=\textbf{\arabic*)}]
    % Questão 1
    \item \textbf{Para o arquivo \texttt{data10.8.dx40}:}
    
    \begin{enumerate}[label=\textbf{1.\arabic*}]
        \item Visualize os dados com o script \texttt{ver.sh} (\texttt{sh ver.sh}).
        
        \item Para realizar a migração é necessário fornecer o valor da velocidade.
        
        \textit{(Nota: dependendo da implementação do algoritmo de migração a velocidade é a "intervalar" ou a "de rms". Da forma como está implementado no SU, a migração de Stolt requer os valores da velocidade de rms).}
        
        Efetue a migração de Stolt, com o script \texttt{mig1.sh} para diferentes valores de velocidade e:
        
        \begin{enumerate}[label=\roman*)]
            \item Explique quais foram os critérios utilizados para decidir qual é a velocidade correta.
            
            --------------------------------------------------------------------------
            
            O objetivo era desfazer os 'laços' entre as hiperboles, fazendo surgir pontos ou linhas contínuas, sem interseção de hipérboles. Para isso, manualmente buscou-se diminuir a velocidade quando as hipérboles que se encontravam tinham formato de 'U' e aumentar em formato de $\cap$. Quando as hipérboles pararam de se encontrar e formaram uma linha contínua, foi definida a velocidade correta. A figura 1 mostra o reusltado obtido.

            \begin{figure}[H]
                \centering
                \includegraphics[width=1\linewidth]{migracao-stolt.png}
                \caption{Migração realizada com velocidade de 1550m/s}
                \label{fig:placeholder}
            \end{figure}
            
            \vspace{1cm}
            \item Forneça o valor da velocidade que escolheu.
            
            --------------------------------------------------------------------------
            
            O valor utilizado foi de $1550 m/s$.
            
            \vspace{1cm} % Espaço para resposta
        \end{enumerate}
    \end{enumerate}

    % Questão 2
    \item \textbf{Analisar influência do comprimento do registro (tempo máximo de aquisição dos dados):}
    
    Compare o resultado da migração anterior com o obtido para o arquivo \texttt{data10.5.dx40}. Explique o porquê das diferenças observadas.

    --------------------------------------------------------------------------

A figura 2 mostra o resultado obtido da migração para o caso dessa questão. 

\begin{figure}[H]
        \centering
        \includegraphics[width=1\linewidth]{igracao-stolt1.png}
        \caption{Migração realizada com velocidade de 1550 m/s.}
        \label{fig:placeholder}
    \end{figure}    

O resultado difere da figura 1, e isso ocorre por conta da diferença do tempo de comprimento de registro. Essa mudança acontece pois a distância e o tempo são relacionados, e a migração tem menos tempo de coleta de dados o que faz ter menos recurso para focar o espalhamento da energia, já que pontos mais profundos têm respostas mais demoradas; com menor tempo, podem não ser obtidos dados suficientes em profundidade, e a energia é espalhada de forma mais inadequada. 
    
    \vspace{1cm} % Espaço para resposta

    % Questão 3
    \item \textbf{Analisar influência do falseamento espacial nos dados.}
    
    \begin{enumerate}[label=\textbf{3.\arabic*}]
        \item Observe o espectro $f-k$ dos arquivos: \texttt{data10.8.dx40} e \texttt{data10.8.dx20}.
        
        Exemplo de comando para gerar a imagem do espectro:
\begin{lstlisting}[language=bash]
suspecfk < data10.8.dx40 | suximage cmap=rgb1 perc=99.9 windowtitle="Espectro f-k" label1="f (Hz)" label2="Kx (1/m) (numero de onda)" title="data10.8.dx40" &
\end{lstlisting}

    --------------------------------------------------------------------------

    As figuras 3 e 4 mostram os espectros.

\begin{figure}[H]
    \centering
    \includegraphics[width=1\linewidth]{f-k-espectro.png}
    \caption{Espectro f-k com dx = 20m}
    \label{fig:placeholder}
\end{figure}

\begin{figure}[H]
    \centering
    \includegraphics[width=1\linewidth]{f-k-espectro1.png}
    \caption{Espectro f-k com dx = 40m}
    \label{fig:placeholder}
\end{figure}

O formato de leque na figura 3 é o adequado, enquanto na figura 4 ele não se mantém, formando uma espécie de losango, por conta do falseamento.

\vspace{1cm}

        \item Efetue a migração para o arquivo \texttt{data10.8.dx20} e explique as diferenças entre o resultado da migração para o arquivo \texttt{data10.8.dx40}.

            --------------------------------------------------------------------------

            As figuras 5 e 6 mostram os reusltados obtidos. Com menor dx, nota-se menos vibrações e ruídos no dados, principalmente nas aréas de mergulho íngrime, onde o serrilhamento é significativamente menor. 

            
            \begin{figure}[H]
                \centering
                \includegraphics[width=1\linewidth]{migracao-stolt2.png}
                \caption{Migração Stolt com dx = 20m}
                \label{fig:placeholder}
            \end{figure}

            \begin{figure}[H]
                \centering
                \includegraphics[width=1\linewidth]{migracao-stolt3.png}
                \caption{Migração Stolt com dx = 40m}
                \label{fig:placeholder}
            \end{figure}

        
        \vspace{1cm} % Espaço para resposta

        \item Efetue a migração para o arquivo \texttt{data20.8.dx40} e explique as diferenças entre o resultado da migração para o arquivo \texttt{data10.8.dx40}.

        --------------------------------------------------------------------------

        As figuras 7 e 8 mostram os resultados obtidos. A diferença entre os resultados é devido ao falseamento espacial nas regiões de mergulho íngrime, pois o dx = 40m é insuficiente para amostrar inclinações de onda de 20Hz, pois quando a frequência dobra, o comprimento de onda cai pela metade. 

        \begin{figure}[H]
            \centering
            \includegraphics[width=1\linewidth]{migracao-stolt4.png}
            \caption{Migração de Stolt com f = 10Hz}
            \label{fig:placeholder}
        \end{figure}

        \begin{figure}[H]
            \centering
            \includegraphics[width=1\linewidth]{migracao-stolt5.png}
            \caption{Migração de Stolt com f = 20Hz}
            \label{fig:placeholder}
        \end{figure}
        
        
        \vspace{1cm} % Espaço para resposta

        \item Qual deverá ser o intervalo entre cdps para os dados apresentados no arquivo \texttt{data20.8.dx40} para que não ocorra falseamento espacial? Explique.

        --------------------------------------------------------------------------

        Um intervalo adequado seria um dx = 20m, pois ao dobrar a frequência, se reduz o comprimento de onda pela metade. Dessa forma, não ocorreria falseamento. 
        
        \vspace{1cm} % Espaço para resposta

        \item Gere o dado sintético para o intervalo escolhido acima utilizando o script \texttt{makedata.sh} e efetue a migração para verificar a qualidade do resultado.
        
        \textbf{Insira a imagem da seção migrada para o dado que foi gerado abaixo:}

            --------------------------------------------------------------------------

            \begin{figure}[H]
                \centering
                \includegraphics[width=1\linewidth]{makedata.png}
                \caption{Dado sintético gerado.}
                \label{fig:placeholder}
            \end{figure}

            \begin{figure}[H]
                \centering
                \includegraphics[width=1\linewidth]{migracao-stolt6.png}
                \caption{Migração de Stolt.}
                \label{fig:placeholder}
            \end{figure}

        
        % \includegraphics[width=0.8\textwidth]{sua_imagem_aqui.png}
        \vspace{1cm} % Espaço reservado para a imagem
    \end{enumerate}
\end{enumerate}

\newpage

% --- PARTE II ---
\section*{PARTE II: Migração no domínio da frequência: exemplo com várias camadas}

A Migração no domínio da frequência de Stolt não aceita variação lateral de velocidades, apenas vertical. Isso não foi uma limitação para o exemplo da Parte I, mesmo para a interface com grande variação lateral de profundidade, pois o modelo era de apenas uma interface.

\begin{enumerate}[label=\textbf{4.\arabic*}]
    \item Avalie o uso da Migração de Stolt na seção empilhada obtida com os dados sintéticos da aula prática de análise de velocidades.
    
    Neste caso, o modelo de velocidades varia lateralmente, mas vamos procurar definir um modelo de velocidades (variação de velocidade apenas com o tempo) que seja o mais representativo possível para os dados utilizados.
    
    Para tal, complete as informações que faltam no comando abaixo:
    \textit{(lembrando que da forma como está implementada no SU, a migração de Stolt requer os valores da velocidade de rms (ou de nmo))}

\begin{lstlisting}[language=bash]
sustolt cdpmin=1 cdpmax=458 dxcdp=25 vmig=____ tmig=____ < arquivo_entrada | suximage ...
\end{lstlisting}

    Informe os valores dos parâmetros \texttt{vmig} e \texttt{tmig} utilizados. Se você avaliar mais que um modelo, informe quais foram os testes realizados.

--------------------------------------------------------------------------

Esse foi o comando utilizado:

sustolt cdpmin=1 cdpmax=458 dxcdp=25 vmig=1500,2000,2500,3000 tmig=0.0,1.0,2.0,3.0 < data20.8.dx20 | suximage perc=99 title="Migracao Stolt V(t)" \&

A figura 9 mostra o dado sintético utilizado. A figura 11 mostra a migração.

\begin{figure}[H]
    \centering
    \includegraphics[width=1\linewidth]{migracao-stolt7.png}
    \caption{Migração de Stolt.}
    \label{fig:placeholder}
\end{figure}

    
    
    

    \item Com base nos seus testes e resultados, comente sobre a eficiência e incertezas do método de Stolt quando aplicado a esses dados.

    --------------------------------------------------------------------------


    O método de Stolt não consegue focar perfeitamente toda a seção ao mesmo tempo. Ele é uma aproximação rápida, mas perde precisão em geologias complexas com forte variação lateral.
    
    \vspace{1cm} % Espaço para resposta
\end{enumerate}

\end{document}