\documentclass[12pt,a4paper]{article}
\usepackage[utf8]{inputenc}
\usepackage[brazil]{babel}
\usepackage{geometry}
\usepackage{amsmath}
\usepackage{amssymb}
\usepackage{graphicx}
\usepackage{enumitem}
\usepackage{titlesec}
\usepackage{fancyhdr}
\usepackage{indentfirst}
\usepackage{float}

% Configurações de página
\geometry{a4paper, left=2.5cm, right=2.5cm, top=2.5cm, bottom=2.5cm}

% Formatação de seções
\titleformat{\section}{\large\bfseries}{\thesection}{1em}{}
\titleformat{\subsection}{\normalsize\bfseries}{\thesubsection}{1em}{}

% Configuração do cabeçalho
\pagestyle{fancy}
\fancyhf{}
\fancyhead[L]{\textbf{AGG0327}}
\fancyhead[R]{\textbf{Atividade 2}}
\fancyfoot[C]{\thepage}

\begin{document}
\thispagestyle{fancy}

\begin{center}
    \textbf{AGG0327 – Sismica de Reflexão \hfill Atividade 1 \hfill Entrega até 11/11/25}
\end{center}

\vspace{0.3cm} % Espaço reduzido

\textbf{Nome:} Gabriel Aparecido das Chagas Silva 14571098

\vspace{0.3cm} % Espaço reduzido

\begin{center}
    \textbf{ANÁLISE DE VELOCIDADES (com o Painel NMO), Correção de NMO e
EMPILHAMENTO: dados sintéticos exemplificando uma linha sísmica}
\end{center}


\section{Questões Parte 1}


\subsection{Execute o script ver.sh (sh ver.sh). Analise a imagem dos dados da linha sísmica e identifique qual dos arquivos está
agrupado em sismogramas de pontos de tiro e qual está agrupado em sismogramas CMP.}




\begin{itemize}
    \item \textbf{arq1 -} Sismograma de pontos de tiro


    \item \textbf{arq2 -} Sismograma CMP
\end{itemize}


\subsection{Utilize o script velan1.sh para realizar a análise de velocidades.
Escolha 3 CDPs, e para cada CDP escolhido anote os valores de t0 e da velocidade que
ajustou o NMO para cada reflexão observada.}



\begin{itemize}
    \item \textbf{CDP 100}: 
    
    $t_0 = 0.38s$, $V_{NMO} = 1600 m/s$

    $t_0 = 0.8s$, $V_{NMO} = 2200 m/s$

    $t_0 = 1.3s$, $V_{NMO} = 2400 m/s$

    \item \textbf{CDP 150}:

    $t_0 = 0.55s$, $V_{NMO} = 1700 m/s$

    $t_0 = 1.1s$, $V_{NMO} = 2300 m/s$

    $t_0 = 1.6s$, $V_{NMO} = 2500 m/s$

    \item \textbf{CDP 200}:

    $t_0 = 0.7s$, $V_{NMO} = 1600 m/s$

    $t_0 = 1.1s$, $V_{NMO} = 2380 m/s$

    $t_0 = 1.5s$, $V_{NMO} = 2600 m/s$
\end{itemize}



\subsection{Faça a correção de NMO e empilhamento. Insira um printscreen da imagem obtida. No comando a seguir, substitua n1,n2,n3 pelos números dos CDPs que você escolheu (em ordem crescente) e escreva os valores dos pares de parâmetros vnmo= e tnmo=, de cada
CDP, respectivamente}



\begin{figure}[H]
    \centering
    \includegraphics[width=1\linewidth]{nmo.png}
    \caption{Figura obtida}
    \label{fig:placeholder}
\end{figure}




\section{Parte 2}


\begin{figure}[H]
    \centering
    \includegraphics[width=1\linewidth]{velocity-stack.png}
    \caption{Empilhamento}
    \label{fig:placeholder}
\end{figure}


\begin{figure}[H]
    \centering
    \includegraphics[width=1\linewidth]{semblance.png}
    \caption{Espectro de velocidade}
    \label{fig:placeholder}
\end{figure}

\end{document}