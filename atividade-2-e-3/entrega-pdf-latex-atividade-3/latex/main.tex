\documentclass[12pt,a4paper]{article}
\usepackage[utf8]{inputenc}
\usepackage[brazil]{babel}
\usepackage{geometry}
\usepackage{amsmath}
\usepackage{amssymb}
\usepackage{graphicx}
\usepackage{enumitem}
\usepackage{titlesec}
\usepackage{fancyhdr}
\usepackage{indentfirst}
\usepackage{float}

% Configurações de página
\geometry{a4paper, left=2.5cm, right=2.5cm, top=2.5cm, bottom=2.5cm}

% Formatação de seções
\titleformat{\section}{\large\bfseries}{\thesection}{1em}{}
\titleformat{\subsection}{\normalsize\bfseries}{\thesubsection}{1em}{}

% Configuração do cabeçalho
\pagestyle{fancy}
\fancyhf{}
\fancyhead[L]{\textbf{AGG0327}}
\fancyhead[R]{\textbf{Atividade 3}}
\fancyfoot[C]{\thepage}

\begin{document}
\thispagestyle{fancy}

\begin{center}
    \textbf{AGG0327 – Sismica de Reflexão \hfill Atividade 3 \hfill Entrega até 25/11/25}
\end{center}

\vspace{0.3cm} % Espaço reduzido

\textbf{Nome:} Gabriel Aparecido das Chagas Silva 14571098

\vspace{0.3cm} % Espaço reduzido

\begin{center}
    \textbf{Sísmica de reflexão - atividade 3}
\end{center}



Seja a equação tempo-distância da onda refletida em uma interface plana e horizontal, em um meio
com velocidade constante (V), dada por:

\begin{equation}
    t^2(X) = t_{0}2 + \frac{X^2}{V^2}
\end{equation}



\section{Questão 1}

\textbf{Qual o significado do termo t0? E qual a relação de t0 com a espessura da camada (lembrando que a reflexão ocorreu na interface
da base da camada)?}\\


\textbf{\textit{Resposta:}}
------------------


O termo $t_0$ é o tempo de offset zero, o tempo vertical de ida e volta na onda na vertical, assumindo a fonte e o refletor no mesmo ponto $x=0$. 

Sabendo que a distância é a velocidade multiplicada pelo tempo, pode-se achar uma relação entre a espessura h e o tempo $t_0$:

\begin{equation}
    2h = V \cdot t_0 \implies h = \frac{V \cdot t_0}{2}
\end{equation}


\section{Questão 2}

\textbf{Para um dado afastamento fonte-receptor (x), a diferença entre o tempo de reflexão t(x) e o tempo da reflexão normal à interface (t(x=0)) é denominada de “Sobretempo Normal”, em inglês o termo empregado é “Normal Move Out” e daí a sigla NMO:}

\begin{equation}
    \Delta t_{NMO} = t(x) - t_0 
\end{equation}


\textbf{Mostre que $\Delta t_{NMO}$ dado pela diferença entre $t(x)$ (eq. 1) e $t_0$, pode ser expresso pelas duas equações a seguir (sendo $V = V_{NMO}$):}


\begin{equation}
    \Delta t_{NMO} = t_0 [ \sqrt{1 + \left( \frac{x}{V_{NMO} t_0} \right)^2} - 1]
\end{equation}


\begin{equation}
    \Delta t_{NMO} = t(x) - \sqrt{t²(x) - \frac{x²}{V²_{NMO}}}
\end{equation} \\

\textbf{\textit{Resposta:}}
------------------

\begin{figure}[H]
    \centering
    \includegraphics[width=1\linewidth]{solucao-questao-2.png}
    \caption{Solução da questão 2 feita a papel.}
    \label{fig:placeholder}
\end{figure}







\section{Questão 3}

\subsection{Camadas com menor velocidade $V$ apresentam maior ou menor $\Delta t_{NMO}$?}

\textbf{\textit{Resposta:}}
------------------


Camadas com menor velocidade apresentam maior $\Delta t_{NMO}$, pois essas grandezas são inversamente proporcionais.


\subsection{Qual dos eventos refletidos na imagem a seguir tem menor velocidade?}

\begin{figure}[H]
    \centering
    \includegraphics[width=1\linewidth]{imagem-3-b.png}
    \caption{Imagem da questão 3b}
    \label{fig:placeholder}
\end{figure}

\textbf{\textit{Resposta:}}
------------------

O evento $t_{0,1}$ tem menor velocidade, pois possui maior $\Delta t_{NMO}$.



\section{Questão 4}

A Correção de NMO corresponde a subtrair o valor do Sobretempo Normal $\Delta t_{NMO}$ de $t(x) $ e assim obter o valor de $t_0$. Utilizando o valor de VNMO correto em qualquer das duas equações acima (4) ou (5), temos que a Correção de NMO  $V_{NMO}$ $t(x) - \Delta t_{NMO} = t_0$ corresponde visualmente à horizontalização da “hipérbole” na imagem do sismograma.

Esboce como o evento hiperbólico de reflexão aparecerá no sismograma se utilizarmos valores
errados de $V_{NMO}$, para as seguintes situações:

\begin{itemize}
    \item a) $V_{NMO} < V_{NMO_{correto}}$
    \item b) $V_{NMO} > V_{NMO_{correto}}$
\end{itemize}


\textbf{\textit{Resposta:}}
------------------

\begin{figure}[H]
    \centering
    \includegraphics[width=1\linewidth]{esboco-questao-4.png}
    \caption{Esboço feito em papel.}
    \label{fig:placeholder}
\end{figure}


\section{Questão 5}

A partir da equação da Velocidade de RMS (Root Mean Square), deduza a equação da Velocidade Intervalar:

\begin{equation}
    V^2_{RMS} = \frac{\sum_{k=1}^{n} V²_k \Delta t_{0_k}}{t_{0_{n}}}
\end{equation}

\begin{equation}
    V_{int} = V_{n} = \sqrt{\frac{V^2_{RMS} t_{0_{n}}  - V^2_{RMS}  t_{0_{n}-1}  }               {t_{0_n}  - t_{0_{n-1}}  }      }
\end{equation}




\textbf{\textit{Resposta:}}
------------------


A velocidade $V_{RMS}$ é a velocidade usada para a correção NMO, que torna a hipérbole horizontal. A velocidade $V_{int}$ é a velocidade real da camada analisada. 



\begin{figure}[H]
    \centering
    \includegraphics[width=1\linewidth]{deducao-questao-5.png}
    \caption{Dedução da equação no papel.}
    \label{fig:placeholder}
\end{figure}




\section{Questão 6}

A partir da observação do sismograma abaixo e dos valores de velocidade RMS obtidos na análise
de velocidades realizada, calcule as espessuras aproximadas das camadas geológicas identificadas.


\begin{figure}[H]
    \centering
    \includegraphics[width=1\linewidth]{imagem-questao-6.png}
    \caption{Sismograma}
    \label{fig:placeholder}
\end{figure}


\textbf{\textit{Resposta:}}
------------------


\begin{itemize}
    \item Camada 1:

    $V_{int_{1}} = V_{RMS_{1}}$ para a primeira camada. Logo, pode-se aplicar diretamente a equação (2):

    $$ h = \frac{V \cdot t_0}{2} = 27.4 m$$


    \item Camada 2:

    Para a segunda camada:

    $$ V_{int} = V_{2} = \sqrt{\frac{V^2_{RMS} t_{0_{2}}  - V^2_{RMS}  t_{0_{2}-1}  }               {t_{0_2}  - t_{0_{2-1}}  }   } = 2244.1 m/s $$  

    Calculando o tempo da onda na camada:

    $$ \Delta t_{0_{2}} = t_{0_{2}} - t_{0_{1}} = 0.041$$

    Aplicando a equação (2):

    $$ h = \frac{V \cdot t_0}{2} = 46.0 m$$

    \item Camada 3:

    Para a terceira camada:

    $$V_{int} = V_{3} = \sqrt{\frac{V^2_{RMS} t_{0_{3}}  - V^2_{RMS}  t_{0_{3}-1}  }               {t_{0_3}  - t_{0_{3-1}}  }      } = 2649.7 m/s $$

    Calculando o tempo da onda na camada:

    $$ \Delta t_{0_{2}} = t_{0_{3}} - t_{0_{2}} = 0.044$$

    Aplicando a equação (2):

    $$ h = \frac{V \cdot t_0}{2} = 58.3 m$$
    


    

    
\end{itemize}



\end{document}